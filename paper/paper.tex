% !TEX program = xelatex

\documentclass[paper=a4,fontsize=12pt,twocolumn]{scrreprt}

\usepackage{fontspec}
\usepackage{polyglossia}
\setmainlanguage[babelshorthands=true]{german}

\usepackage[autostyle,german=quotes]{csquotes}
\usepackage[autostyle]{csquotes}

\usepackage[toc,page]{appendix}

\usepackage{array}
\usepackage{multirow}
\usepackage{multicol}
\usepackage{hyperref}
\usepackage{url}
\usepackage{listings}
\usepackage{xcolor}

\usepackage{graphicx}

\usepackage{kantlipsum}

\usepackage[
    backend=biber,
    style=numeric,
    citestyle=authoryear,
]{biblatex}

\addbibresource{./literature.bib}
\graphicspath{ {./images/} }

\usepackage{amsmath}

\title{Project::Foo Abschlussbericht}
\author{Maximilian Bauregger \and Leonard Caanitz \and Lennart Clasmeier \and Patrizio Ferrara \and Luca Müller \and Frederic Voigt}
\date{\today}

\begin{document}

\maketitle

\tableofcontents

\section*{Abstract}

\begin{itemize}
    \item Was wurde wie erreicht?
    \item Wo pflegt sich das in den den aktuellen stand der Forschung ein - wir haben also eine reduzierte ARM Version gebaut
\end{itemize}
\kant[1]


\chapter{Einführung}
\begin{itemize}
    \item Hier wird das Projekt kurz vorgestellt
    \item Was haben wir gemacht? Was haben wir vor? Welche Tools wurden benutzt?
    \item Welche Schwierigkeiten haben sich abgezeichnet?
    \item welche Struktur hat der Vortrag - Welche Informationen können die Leser\_Innen in welchen Kapiteln erwarten.
\end{itemize}
\kant[2-3]

(Max)

\chapter{Planungsphase}
Zuerst haben wir in einem Python Sketch die Struktur des Prozessors definiert.
Danach wird gemeinsam über den möglichen Befehlssatz diskutiert.
Nachdem diese Dinge geklärt sind, haben sich die Gruppen in den jeweiligen Bereich Software und Hardware aufgeteilt, um so das Besprochene zu realisieren.
Die programmierte Hardware und Software soll zusammen verbunden, getestet und eventuelle Fehler behoben werden, bis keine Fehler vorhanden sind.
Falls noch Zeit übrig sein sollte, sind Erweiterungen des Befehlssatzes, sowie weitere Datentypen mögliche Punkte, um den Prozessor qualitativ zu verbessern.
Es sind einige Fragestellungen und Diskussionspunkte aufgekommen, die im Nachfolgenden genauer besprochen werden sollen.
Darunter finden sich auch Punkte oder Konzepte, die im Diskussionsprozess ausgeschlossen und nicht im finalen Projekt manifestiert wurden.
Diese wollen wir dennoch beleuchten.

\kant[5]

(Wer?)


%%TODO Eingführung ins Chapter Instruction Set sschreiben, was wird in welcher tiefe vorgetsellt was passiert in welchen kapitel später mit deiesen infromationen, wozu ist das wichtig?

\section{Wortlänge und Adressmaschine}

Bei der Adressmaschine wurde sich für eine 3 Adressmaschine entscheiden.
Die 0- oder 1- Adressmaschine wurde direkt ausgeschlossen, da diese zu umständlich und nicht zielführend gewesen wären.
Die 3-Adress Maschine hat zwar keine funktionalen Vorteile, gegenüber der 2-Adressmaschine, aber der Aufwand, komplexere Abläufe zu realisieren ist geringer.
Ein „add“-Befehl kann sein Ergebnis beispielsweise direkt an eine dritte Adresse schreiben und ist in einem Befehl realisierbar.
So werden umständliche Kopiervorgänge gespart.
Ein weiteres Argument für die 3-Adressmaschine war die Reduzierung von Abhängigkeiten.
Die Zahl an NOOPs, die für eine korrekte Ausführung benötigt werden, wird minimiert.
Das ist sowohl schneller, als auch einfacher in der Handhabung.

Da hier Performance nicht relevant für die Entwicklung war, wurde - konsequenterweise - auch die Bitzahl als irrelevant eingestuft.
Dadurch ist auch das Einsparen von Bits, im Befehlswort durch eine 2-Adressmaschine nicht nötig.

Bei der Bit-Zahl wurde sich für 32 entschieden. 32 Bit Wortlänge reicht für alle Befehle und werden nie ganz ausgeschöpft. Darum wäre auch eine längere Bitzahl überflüssig gewesen.

(freddy)



\section{32-Bit-Code}

In den folgenden Kapitel wird die Struktur der 32-Bit-Codes behandelt. Es wird beschrieben wie der Code Unterteilt wurde und welche Denkprozesse dabei durchlaufen worden sind. Angefangen von der Unterteilung der Bits, über die Auswahl des Befehlssatzes und des daraus resultierenden Opcodes bis hin zu den Spezialfällen die aus den vorherige Entscheidungen heraus entstanden sind. 

\subsection{Struktur ( Opcode- / Register- / Immediate- Bit-Länge )}

Bei den Überlegungen wie die Struktur des Codes aussehen soll kamen die folgenden Ergebnisse: 
\begin{itemize}
    \item Die ersten Bits stehen für den Opcode. Die Länge wurde in Abhängigkeit an der Anzahl der Befehle auf 5 Bits gewählt. Diese ermöglichen das ergänzen weiterer Befehle da nicht alle Opcodes zugewiesen sind. 
    \item Das darauf folgende Bit signalisiert ob es sich beim letzten Operanden um ein Immediate-Wert handelt. Näheres wird im Kapitel Immediate-Bit erläutert.
    \item Die restlichen Bits sind für die Operanden reserviert. Dabei wurde sich auf 16-Bit langen Immediate-Wert geeinigt. Dieser erleichtert für die Softwareprogrammierung das darstellen größerer Zahlen (bis zu 32Bit) mit insgesamt 3 Befehlen (Zuweisung in ein Register der ersten 16 Bits, 16 Shift des Registers, Zuweisung der letzten 16 Bits in das Register). 
    Durch die 5 Bits des Opcodes, 1 Bit des Immediete-Bits und 16-Bits des möglichen Immediatewertes, können aus Platzgründen nur noch 5 Bits für die Register zugewiesen werden. Dies ergibt eine Gesamtzahl von 32-Registern. 
    
\end{itemize}

\begin{figure}[h]
\centering
% \includegraphics{Figure 3.2.1}
\caption{Bild der Struktur}
\end{figure}

\subsection{Befehlssatz}

Um einen Überblick über mögliche Befehlssätze zu bekommen, wurde sich an den ARM Befehlssatz orientiert\footnotemark.
\footnotetext{//(*link zum ARM Befehlssatz)//}
Dieser wurde noch reduziert, da nicht alle Befehle für das Projekt relevant sind.
Bei der Reduzierung wurden die Befehle in 3 Kategorien unterteilt:
\begin{itemize}
    \item \enquote{must-have}:
    Das Minimum das nach Erachtens der Gruppe der Prozessor haben muss um die gestellten Anforderungen zu erfüllen.
    \item \enquote{nice-to-have}:
    Funktionen die bei übrig gebliebener Zeit implementiert werden können.
    \item \enquote{irrelevant}:
    Zu komplexe oder für das Projekt irrelevante Funktionen.
\end{itemize}
Für den vorläufigen Befehlssatzes wurde sich zunächst auf die \enquote{must-have} Befehle fokusiert. 
Der somit entstandene Befehlssatz musste nun einem Opcode zugewiesen werden. Hierbei musste entschieden werden, ob ein Opcode bei unterschiedlichen Typen von Parametern(Immediate-Wert bzw. Register) nicht gleich ist. Dies hätte zur Folge, dass bei Befehlen, die einen Immediete-Wert zulassen, ein eigener Befehl erstellt werden müsste. Wie z.B bei einer Addition; das \texttt{ADD} müsste mit einem \texttt{ADDI} ergänzt werden. Eine andere Möglichkeit ist die Unterscheidung nicht im Opcode stattfinden zu lassen, sondern diese auszulagern.


\subsection{Immediate-Bit}

Das Immediate-Bit ist eine Mögliche Lösung um diese Auslagerung zu realisieren. In diesem speziellen Fall wäre das ein Bit das sich nach dem Opcode befindet. Es ist \enquote{0}, wenn es sich in de Zuweisungen nur um Register handelt und \enquote{1}, wenn eine Zuweisung ein Immediate-Wert ist. Somit wird auch schon im 32-Bit-Code ersichtlich, welche Typen von Parameter zugewiesen werden.
Eine Auslagerung und das hinzufügen eines zusätzlichen Befehles im Befehlssatzes würden bei diesem Projekt schlussendlich zum selben 32-Bit-Code führen.
Vorteile einer Auslagerung ist die geringere Anzahl an Befehle die benötigt wird.
Dadurch wird die Softwareprogrammierung vereinfacht, da sie zwischen einer direkten und indirekten Zuweisung nicht unterscheiden muss.
Aus diesen Gründen wurde beschlossen, die Unterscheidung nicht im Opcode zu behandel, sondern ein Bit in der 32-Bit-Anweisung als \enquote{Immediate-Bit} zu deklariert.


\subsection{Zuweisung des Opcodes}

Die ausgewählten Befehle wurden in 3 Kategorien unterteilt, die sich in der Anzahl der angegebenen Parameter unterscheiden. Um diese Unterscheidung auch im Opcode erkenntlich zu zeigen, haben alle 3 Kategorien (ONEOP, TWOOP,THREEOP) unterschiedliche erste 2 Bits in ihrem Opcode. Sodass eine konsistente Regel ihrer Layout erstellt werden kann.

\begin{figure}[h]
\centering
% \includegraphics{Figure 3.2.1}
\caption{Bild der mögliche Layouts}
\end{figure}

Dies dient vor allem der Übersicht sowie einer einfacheren Decodierung, wie im Kapitel Decoder beschriebe ist. 


\subsection{Diskussion bei Befehlen}

Bei den Unterteilungen der Befehle in die Anzahl der Parameter kam es bei 3 Befehlen zu Diskussionen.
Betroffen waren folgede Befehle: der \enquote{Jump}-Befehl, der \enquote{Store}-Befehl und der \enquote{Load}-Befehl. 
Bei der Diskussion ging es primär um die Frage, ob relative Zuweisungen möglich sein sollen. Das heißt in diesen speziellen Fällen, ob z.B. relative Sprünge möglich sein sollen. 


%  wieviele Operanten haben die Befehle. Also ist Jump ein oneop oder threeop.
% "Das relative Sprünge möglich sind"




(paddy)

\chapter{Hardware}

% TODO Missing: Aufstellung von Problemen

\begin{itemize}
    \item VHDL
    \begin{itemize}
        \item Struktur der Komponenten
        \item Zu "wörtliche" Umsetztung des Python-Sketches (alles in entities verpackt, wir wussten nicht wie Register funktionieren)
        \item Probleme mit dem reinen Entity - Architecture - Ansatz
        \item Parallelität und sequenzielle Abläufe
        \item Clock
        \item Pipeline Nutzung
        \item Finaler Entwurf
    \end{itemize}
    \item Tests
    \item Nice additions
\end{itemize}

Dieses Kapitel behandelt die Umsetzung unseres Mikroprozessors in Hardwarekomponenten in VHDL.
Zeitlich betrachtet schließt dieses Kapitel also an die Arbeiten des Hardware Teams nach der gemeinsamen Planungsphase des gesamten Teams an.
%eventeull als zeitliche einleitung für software übernehmen 
In dieser vorangegangenen Planungsphase einigte sich die Gruppe auf zentrale Orientierungspunkte, welche die Arbeit des gesamten Teams strukturieren.
Diese wurde bereits in Kapitel (?!) vorgestellt.
Dazu zählen fundamentale Konzepte, wie die Wortbreite und der entlehnte ARM-Befehlssatz (siehe Anhang).
% Referenzen schönmachen
Die durch das Team geleistete Vorbereitung stellt uns einen detaillierten Leitfaden zur Umsetzung bereit.
Auf dem Papier ist bereits an diesem Punkt ein Prozessor in Python Pseudocode entstanden, der zur Orientierung dienen sollte (siehe Anhang).
%praktisch dann raus
Praktisch musste dann nur der Python Code in funktionierenden VHDL Code umgewandelt werden.
Das stellte die erste Herausforderung da.
Von uns hatte niemand Erfahrung auf dem Gebiet von Hardware-Beschreibungssprachen.
%Nicht alles ist eine große AUfgeb oder Herausfoderung. Sagen worum es geht!
Intern hat sich das Hardware-Team darauf geeinigt sich mindestens einmal pro Woche zu Treffen um erarbeitete Lösungen zu besprechen, das weitere Vorgehen zu Planen und die Arbeit aufzuteilen.
%so etwas auch für software Einleitung vorgehen dies das 

Im Folgenden soll einerseits der Prozess aus Sicht des Hardware Teams, der schlussendlich zu einer funktionierenden Lösung geführt hat, beschrieben werden.
Andererseits soll die von uns erarbeitete Lösung vorgestellt und diskutiert werden.
% Der folgende Ttext gibt einen einblick in den Prozess und stelltt die Inhaltichen schrittte vor.

%%Im Fokus stehen dabei einerseits der .

(luca)

\section{Vorgehen}

Als Hardware-Team haben wir uns intern darauf geeinigt uns einmal pro Woche zu treffen.
Hier wurden selbstständig erarbeitete Lösungen, Probleme und die Arbeitsteilung besprochen.
%steht genau so in der Einleitung
Da für alle beteiligten Programmieren in VHDL neu war, haben wir uns in der frühen Phase des Projekts bzw. unmittelbar am Anfang auf ein Drei-Phasen-Programm geeinigt.
In der ersten Phase haben alle Teammitglieder sich allgemein mit VHDL vertraut gemacht.
Wie blieb den einzelnen Gruppenmitgliedern überlassen.
Vielversprechendes Material wurde im GitLab gesammelt.
Darüber hinaus wurden Schwerpunkte vergeben zu denen unter anderem Entities, Architectures, Prozesse und mögliche Umsetzungen von Pipelining gehörten.
Als Ergebnis dieser ersten Recherchen haben wir uns für einen strengen Entity Architecture Ansatz entschieden.
% was ist ein "strenger entity architecture" ansatz genau?
%Außerdem haben wir uns nicht dafür entschieden sondern wir dachten das wäre vlt nne gute Idee. Wir haben halt versucht VHDL zu verstehen. Was kapslett man was kommt in eigene DAteien Die sachen mit den Imports und scooping und namespaces- So was vlt ncohmal stärken. Der kerngfehler: Jede reale Harware komponente braucht eine entsprehung als Entity Arc
Es erschien uns zu diesem Zeitpunkt logisch jeder Stage im Prozessor seine eigene Entity und Architecture zu geben.
Die in den Entities definierten Schnittstellen sollten die Kommunikation der einzelnden Stages untereinander gewährleisten, während die jeweilige Architectures die gewünschte Datenverarbeitung übernähme.
In der zweiten Phase wurden die Stages aufgeteilt, sodass jeder mindestens eine Komponente in VHDL programmiert hat.
Dabei stand nicht im Vordergrund eine perfekte Lösung zu erarbeiten.
Viel mehr sollte ein Rahmen konstruiert werden.
In der dritten Phase würden dann die individuell erstellten Komponenten - so unsere Vorstellung - nach Baukastenprinzip zusammengesetzt und so lange verbessert werden bis ein lauffähiger Prototyp entsteht. %ENTITY = BAUTEIL = FALSCH <- schön aubauen
Bei diesem Prozess haben wir zunächst noch einmal von unserem Entwurf abstrahiert und in der Umsetzung reduziert.
So war zum Beispiel Pipelining kein Teil dieser Lösung. %STARK machen dass das einn falsches darüber nachdnken war PIPELINING EXISTS
%klären inwie weit pipelining doch schon teil der lösung war 
%piplininggg war schon imemr teiol der löszung, richtiger wäre wir hatten keine ahnung und haben pipeline register eher aussgespart bis wir dann wussten wie das ging
Der Fokus lag darauf lauffähigen VHDL-Code zu produzieren.

\begin{figure}
    \centering
    % \includegraphics{}
    \caption{Hier kommt vielleicht nochmal ne Grafik hin}
    \label{fig:my_label}
\end{figure}

Dieser strikte Entity-Archittecture-Ansatz führte jedoch nicht zu einer Lösung und so ist die dritte Phase in der Form nie eingetreten.
Der von uns verfolgte Ansatz führte aber insofern zum Erfolg als dass alle Teammitglieder erste theoretische und praktische Grundlagen in VHDL erlangt haben.
Dadurch waren alle Teammitglieder in der Lage sich an Diskussionen zu beteiligen.
Durch die individuelle Spezialierung konnten wir uns gegenseitig Hilfestellung leisten und es fielen Probleme auf, die andern zunächst verborgen blieben.

Zeitlich gesehen stellt sich dieser Punkt als Halbzeit des Projekts dar.
%DAs nahm mehr als die hälfte der verwendetten zeitt ein
Bis dahin haben wir E/A für CPU, ALU, Decoder, Sign-Ext, PC und Pipeline Register geschrieben.
In der Diskussion um Flag Handling wurde uns klar, dass wir grundlegende Konzepte in VHDL nicht oder falsch verstanden haben.
Am prominentesten ist das Konzept von Nebenläufigkeit in VHDl zu nennen und die Frage danach wie genau Prozesse funktionieren.
Der Umstand, dass Prozesse nebenläufig sind, die Statements innerhalb aber sequenziell abgearbeitet werden, führte im Folgenden immer wieder zu Problemen {\color{red}ZU WELCHEN EIGENTILICH??}
In der Retrospektive könnte man sagen: \enquote{Wir haben zu funktional gedacht}.

%Erzählperspektive


% Vielleicht kommt das auch einfach raus oder woanders hin?
%Es klingt vielleicht naiv aber die Erkenntnisse die Signale so zu betrachteten wie elektrische Leitungen in Silikon(Silizium ;) ) war ein gedanklicher Wendepunkt in unserer Arbeit.  {\color{red}Das ist irgendwie noch sehr holprig und belletristisch, da muss man nochmal ran}

%Weil wir einfach die Signale durch pusten von Stage zu ssttaaggee durchpusten wollten und die Nebenläufigkeit der Komponenten total außer acht gelassen haben bzw. uns darüber nicht im klaren waren,

Bis dato existierte noch kein Speicher und Pipelining war auch noch nicht implementiert.
Wir mussten unseren Ansatz noch einmal neu denken. Genauer: neu darüber nachdenken, wie sich unser Ansatz in VHDL sinnvoll umsetzten lässt.
Im folgenden Kapitel sollen die Probleme und die anschließend daraus gezogenen Schlüsse kurz Diskutiert werden um im Anschluss in [Kapitel BLANK] den letztendlichen Entwurf vorzustellen. 

\subsection{Vorläufige Problemsammlung}
% feels like a leap - vllt anderer titel?

%TEILL DAS IN DIE KAPITEL AUF


Hier werden einige Probleme unserer ersten besprochen und Lösungsansätze kurz besprochen werden. Dabei soll davon abgesehen werden jede kleine Änderung im Code abzuarbeiten welche die letztendliche Lösung zum Resultat hatten. Viel mehr soll ein Eindruck davon vermittelt werden welche entscheidenden Impulse diese Auseinandersetzung geliefert um im nächsten Kapitel unsere Lösung detailliert vorzustellen.
Wir mussten Feststellen, dass immer wieder undefinierte Werte in unseren Registern aufgetaucht sind oder Programme falsche Ergebnisse lieferten.
Dass lag daran, das falsche Werte propagiert wurden und/oder gleichzeitige Zugriffe auf Signale stattfanden.
Hier rächte sich ein wenig unsere Idee zunächst einen lauffähiges Programm zu produzieren bevor wir uns an die Feinheiten machen wollten.




\begin{itemize}
    \item 
    %Gedanklich sind wir aber schon von Pipelining ausgegangen und haben die Parallelität der Komponenten außer Acht gelassen.
    Signale auf denen Werten an lagen wurden weiter propagiert und von den nachkommenden Stages verarbeitetet, änderte sich dann das Signal änderten sich auch die Werte in nachfolgenden Prozessen - auch wenn das nicht gewünscht wurde.
    Wir haben daran gelernt die Signale wie elektrische Leitungen zu betrachten, eben so wie sie nachher auch synthetisiert werden.
    Hier wird auch deutlich was damit gemeint ist wenn wir: "zu funktional gedacht haben".
    Die Lösung war die Signale zu takten indem die Inputs der Stages von der Clock abhängig gemacht gemacht wurden. Die so entstandenen Register implementierten  endlich das Pipelining implementiert.
    \item Alle Signale sollten nach Möglichkeit durch alle Stages geschleift werden, ob sie intern verarbeitetet würden oder nicht. Das führte zu zwei Problemen mit dem PC-Wert. Das erste Problem betraf die Initialisierung des Pc-Wertes, der mit jedem neuen Takt hochgezählt werden sollte um in der nächsten Fetch Vorgang den dem Korrekten Wert anzunehmen. Dieses Problem  haben wir zunächst durch entsprechendes Initial setzen passender Werte gelöst(4,3,2,1). Damit liefen bereits erste einfache Programme wie(?!?!?!?). Dieser Ansatz scheiterte aber sobald das Programm Sprünge verlangte. Zwar wurde die Sprungadresse richtig gesetzt, da es keinen Flush für die Pipeline gibt wurden die nun obsoleten weiter propagiert. %Welche obsoleten?
    Das Problem lösten wir indem wir den pc unabhängig von den Stages durch ein eigenes Registern speicherten und (was nochmal?)
    \item Ein ähnliches Problem ergab sich mit den Flags. Diese sollten auch durch die Stages geschleift werden. Da sie so aber immer auch als Input Wert propagiert wurden führt das ggf zu unerwünschten Verhalten. Wir hätten also einen Mechanismus einführen müssen um zu Testen ob der richtige Wert anliegt. Wir haben uns entschieden die Flags ebenfalls als unabhängige Signale zu setzten. Diese werden nun stetig berechnet und liegen als Werte an. Stages bzw. Operationen welche die die Flags benötigen greifen dann auf sie zu wenn sie diese benötigen(wie nochmal genau)
\end{itemize}

% Was will dieses Kapitel genau?
    %Vorerst Probleme sammeln die Bei der Arbeit aufgetreten sinnd
% Probleme konkret benennen?
% Inwieweit findet sich das unten wieder
%Eindampfen
% später drüber reden (luca und max?)


(luca)

\section{Struktur oder Setup oder wie das aussieht}
Struktur und Kommunikation -
% Zusammenhänge von componenten
% Wie fließen daten durch den prozessor
% componenten, entities, architectures - vhdl setup
%Datentypen
%Packages und Components - Subtypen

%mpure muss in architectture kann aber auf werte außerhalb der funkttion zugreifen solnag sie in der architectue ssind

Die CPU ist in fünf \textit{Stages} unterteilt.
Befehle \enquote{fließen} durch diese Stages hindurch.
Jede dieser Stages besitzt Input- und Outputsignale.
Die Inputsignale einer jeden Stage werden \textit{clocked} beschrieben.
Sie werden somit als Register synthetisiert.
Der Prozess, der dies veranlasst, besitzt das Suffix \texttt{\_pipeline}.
Die Outputs einer jeden Stage sind nicht geclocked und somit nur \enquote{einfache} Datenleitungen.

\begin{description}
  \item[Fetch]
  Die Fetch Stage ist recht simpel.
  In ihr wird die Input Instruction clocked gelesen und auf den Output gelegt.
  % TODO really? it seems like, its outputs are never used.
  % TODO remove fetch
  \item[Instruction Decode]
  Im Instruction Decode wird die gelesene Instruktion dem Decoder präsentiert und die Outputs des Decoders genutzt um entsprechende Register und Flags zu lesen.
  \item[Execute]
  Die Execute Stage legt entsprechende Signale an die ALU-Komponente an und gibt das Resultat aus.
  \item[Memory Access]
  Hier passieren zwei verschiedene Dinge.
  Zum Einen wird entschieden, ob gesprungen wird oder nicht.
  Dies wird als Outputsignal der nächsten Stage präsentiert.
  Zum Anderen wird das \texttt{data\_addr} Signal entsprechend geschaltet, sofern der Opcode ein \texttt{STR} oder \texttt{LDR} ist, und Daten entweder geschrieben oder gelesen werden.
  % Irgendwie komisch vllt zwei sätze
  \item[Write Back]
  Im Write Back passieren wiederum zwei Dinge parallel.
  Auf Basis des \texttt{will\_jump} Signals wird entweder der Program Counter (PC) um eins erhöht oder auf das Resultat der Memory Access Stage gesetzt.
  Sofern der bearbeitete Befehl in ein Register schreiben soll, wird dies ebenfalls getan.
\end{description}

\subsection{Setup}

Die CPU ist in drei VHDL Entities unterteilt: Die \texttt{cpu} selbst, die \texttt{alu} und der \texttt{decoder}.
Die gesamte Pipelinelogik ist in der \texttt{cpu} Entity untergebracht.
Sie besitzt nach außen Adress- und Datenleitungen für die Anbindung der Speicher\footnotemark und einen \texttt{clk} Input.
\footnotetext{Unser Design geht von zwei getrennten Speichern für Daten und Instruktionen aus.}
Die dazugehörige Architektur \texttt{cpu\_arc} nutzt die \texttt{decoder} und \texttt{alu} Komponenten intern.
Es wird außerdem ein \texttt{cpu\_pkg} bereitgestellt, dass die \texttt{cpu} Komponente und oft verwendete Typen enthält.

Die \texttt{alu} Entity ist recht simpel gehalten.
Sie nimmt eine \enquote{Rechenaufgabe} mit zwei Operanden entgegen und gibt das Ergebnis sowie eventuelle Flags nach außen.
Da die ALU keinen internen Zustand hat, besitzt sie auch kein \texttt{clk} Signal.
Neben der \texttt{alu} Entity selbst beinhaltet das \texttt{alu\_pkg} die speziellen ALU Opcodes und die \texttt{alu} Component.

Der Decoder ist ähnlich zur ALU gehalten.
Die \texttt{decoder} Entity erhält nur eine Instruktion als Input.
Sie ist ebenso nicht \texttt{clk}-abhängig.
Die Outputs des Decoders bestimmen, welche Operation die ALU ausführt und welche Register gelesen werden.
Im \texttt{decoder\_pkg} sind die \texttt{decoder} Component und alle Opcodes definiert.

% Component begriff erläutern? (Am besten vorher)

(max)

\section{Input / Output}

Die CPU benötigt natürlich noch Speicher für Daten und Instruktionen um ein Programm ausführen zu können.
% Colloquial? "naturlich noch" weg?
Dieser wird in der \texttt{processor} Entity mit der CPU verbunden.
Es wurde die \texttt{sram2} Entity, welche uns von Andreas Mäder\footnotemark zur Verfügung gestellt worden ist, verwendet.
\footnotetext{Guter Mann. Kudos dafür! *bussy* und Herzchen}
Im \texttt{processor} werden zwei Instanzen dieses Speichers erzeugt - eine für Daten, eine für Instruktionen.
Der Adress-Input dieser Speicher ist mit den jeweiligen Adress-Outputs der CPU verbunden.
Da die Speicher nicht mit vollen 32 bit Adressen operieren, werden die Output-Adressen der CPU \enquote{beschnitten}.
Es ist somit nicht möglich den vollen 32 bit Adressraum zu nutzen.
So wird aber nicht der gesamte 32 bit Raum beim simulieren alloziert.

Des weiteren stellt der \texttt{processor} ein \texttt{clk} Signal für Speicher und CPU bereit.

(max)

\section{Register}

Neben vielen internen Registern, wie zB. den Pipelineregistern bestitzt die CPU 32 \enquote{Datenregister}.
Diese können von laufenden Programmen ausgelesen und meist auch beschrieben werden, um zB. Rechenergebnisse zwischenzuspeichern, ohne in den Speicher schreiben zu müssen. 
Diese sind mit fünfbittigen ($2^5 = 32$) Adressen zugreifbar.
Einige dieser Register dienen einen speziellen Zweck - sogenannte \enquote{Special purpose} Register - und unterliegen gewissen Restriktionen.
\begin{description}
    \item[Zero]
        Das Zero Register R0 hat immer den Wert $0$.
        Es ist in dem Sinne kein Register sondern 32 dauerhaft auf 0 gezogene Bits und daher auch nicht beschreibbar.
    \item[Flags]
        Um Flags wie compare, overflow und carry zu speichern, wird das erste Register R1 verwendet.
        Berechnet die ALU eine Operation, welche eine Änderung der Flags zur Folge haben könnte (COMPEQ, COMPGT, ADD, ADC, SUB, SBC), wird das entsprechende Bit in der Write Back stage im Flags register gesetzt.
        Das Register ist frei beschreibbar.
    \item[Link]
        Im Falle eines Sprunges, wird der aktuelle PC wert in das link Register geschrieben.
        Dies ermöglicht es, wenn zu einer Subroutine gesprungen wurde, wieder zum ursprünglichen Programm zurückzukehren.
        Das Register verhält sich sonst wie ein generisches Datenregister.
        Es ist vor allem Aufgabe des Programmes, sofern zu einer weiteren Subroutine gesprungen werden soll, sicherzustellen, dass der Zustand des ursprünglichen Link-Registers nicht verloren geht. 
    \item[PC]
        Das PC Register gehört - ähnlich zum Zero-Register - in dem Sinne nicht zu den anderen dem Programm zugänglichen Registern.
        Es ist nicht in der Registerbank verortet sondern steht für sich.
        Der \texttt{instr\_addr} Ausgang der CPU ist fest mit dem PC Register verbunden.
        Um Programmen das Auslesen des PC Registers zu ermöglichen, ist es als R31 addressierbar. Schreiboperation werden jedoch ignoriert.
        Hierfür muss dediziert der \texttt{JMP} Befehl verwendet werden.
\end{description}

Die Implementation der Datenregister stellte uns vor einige Schwierigkeiten.
Uns war nicht bewusst wie genau ein Register in VDHL zu formulieren war.
Der erste Ansatz eine eigene Entity pro Register zu schaffen wurde schnell verworfen.
Als uns klar wurde, dass Register letztlich nur \enquote{geclockt} beschriebene Signale sind, schufen wir die Registerbank - letztlich nur ein Array mit 32 Feldern vom  Typ \texttt{std_logic_vector} mit einer Länge von 32 bit.
Zugriffe auf diese Datenstruktur führten immer wieder zu undefinierten Werten.
Wie uns später klar wurde ist es keine gute Idee konstante Werte an Register anzulegen oder zeitgleich zum gleichen Zeitpunkt zweifach in das Register zu schreiben.
Die Lösung lag darin, alle Schreibzugriffe in einem Prozess im Writeback - \texttt{wback_pipeline} - abzuhandeln.
Das PC Register wurde in dem Zuge auch aus der Registerbank ausgelößt.
Es wird nun durch den - und nur den - \texttt{wback_pc_set} Prozess im Writeback beschrieben.

(max)

\section{PC Handling und Sprünge}

%was ist next sequence pc
Um den richtigen Befehl aus dem Instruction-Speicher zu laden wird ein Register namens Programm Counter (PC) verwendet, indem die aktuelle Adresse steht. Der PC wird in der fetch-stage verwendet um Befehle aus dem Speicher laden zu können. Der PC muss in jedem Clock Zyklus um Eins inkrementiert werden, um bei der nächsten fetch-stage Ausführung den nächsten Befehl anzuvisieren. 

Die ursprüngliche Implementation hatte in jeder Pipelinestage einen next-sequence PC Wert gespeichert, der den PC Wert enthielt der gesetzt sein soll, sobald der indizierte Befehl abgearbeitet werden soll. Dieser wurde in der fetch-stage berechnet und in der write-back stage genutzt,um die Speicheradressen zu setzten. 
Der next-sequence PC wurde aufgegeben, da es einige Schwierigkeiten gab:

Die erste Implementation war fehlerhaft. Das Durchlaufen eines Befehls durch die Pipeline dauert vier schritte.Der next-sequence PC wurde aber nur um eins erhöht. Wenn dieser befehl im memory access landet und genutzt wird um die Instruktionsspeicheradresse zu setzten, sind schon vier Zyklen vergangen, seitdem in der fetch-stage der next-sequence Pc gesetzt ist. 

Eine weitere Schwierigkeit ergibt sich aus den fehlenden Initial-Zuständen.
Jede Komponente, inklusive des next sequence PC. ist undefiniert, bei Simulationsstart.Auch in der Realität kann nicht angenommen werden, dass alle Werte auf 0 gesetzt sind. Dies führt insbeossndere beim next-sequence PC Handling zu schwierigkeiten, da in jedem Clock zyklus der Wert im next-sequence PC im memory access als Adresse genutzt wird. Ist dieser undefiniert, erhält das Instructions-memory einen undefinierten Wert. Weiterhin muss daraufgeachtet werden, wie die richtigen Initialwerte gesetzt sind: Wäre alles mit 0 initialisiert, würde vier mal der nullte Befehl geladen werden. Der erste Befehl wird auf 1 gesetzt und die anderen drei werden wieder mit 0 geladen. Die fetch-stage nimmt sich weierthin den nächsten Befehl.

%Bild reinpacken

Die Jumps haben zu weiteren Problem geführt.
Die Motivation zur Implementation des next-sequence Pcs war die vermutung, dass Sprünge damit einfach realisierbar gewesen wären. Ein befehl trägt, beim Durchlauf durch die Pipelinestages, den Wert des PCs mit sich. Diesen Wert umschreiben, wenn er ein Sprung ist, schien zunächst einfach. So mit wäre eine externe Sprung Logik nicht nötig. 

Angenommen ein Jump-befehl läuft durch die  Pipeline. Der next-sequence PC liegt bei dem aktuellen Pc Wert plus 4, sobald dieser Befehl in der fetch-stage ist. Läuft dieser Jump-befehl jetzt weiter und kommt in der memory-access stage an, ändert diese den next-sequence PC Wert auf die Sprungadresse. Das sorgt dafür, dass der nächste Befehl, den die fetch-stage laden wird, der ist, auf den gesprungen wird. Die problematik gibt sich nun aber daraus, dass der Befehl direkt nach dem Jump (der eine Stufe weietr oben liegt) von dem Sprung verhalten nicht geändert wird. Einen Prozessortakt weiter und der nachfolgende Befehl im memory-acess landet, ist der next-sequence Pc von diesem Befehl nicht vom Sprung berührt und wird direkt wieder für den Instrction-Speicher verwendet. In der fetch-stage wird der nächste Befehl mit diesem PC Wert geladen. 
Das Ganze ist offensichtlich fehlerhaft und wurde schlussendlich auch verworfen.
%reihenfolge 4, 100, 5, 6, 104

Der PC Wert hatte auch noch zu Problemen geführt, da er ursprünglich ein normales Register war. Das beschreiben von Registern gleichzeitig ( Pc Register und Ziel-Register des Befehls) führte zu Fehlern.
Der PC ist jetzt aus der Registerbank ausgelagert.
Das verhinderte die undefinierten Werte in den zwei Register(Banken).

Die finale Implementationm hat keine next sequence PC Werte mehr, sondern nur noch einen PC Register außerhalb der Pipeline.
Dies ist auch getrennt von der Registerbank.
Im write back wird dieser immer um 1 erhöht, sofern nicht gesprungen wird.
Wird im memory-access ein Sprung festgestellt, wird ein Will-Jump Signal vom memory access zum wirte back auf 1 gestellt. Dies veranlasst die writeback stage nicht den PC-Wert um 1 zu erhöhen, sondern den Target Wert als neuen PC-Wert zu setzen. 
Die gesamte PC-Logik ist somit im write back.
Durch die Anpassungen ist die gesamte fetch stage obsolet geworden. Die  Instruction Decode stage ließt direkt den Input vom instruction memory. 

(freddy)

\section{Decoder}

Der Decoder ist als eine eigene VHDL-Entity definiert. Seine Aufgabe es ist den 32-Bit-Code von dem Instructionmemory zu interpretieren, dessen Information zu filtern und diese den entsprechenden Outputs zuzuweisen.
Wie im Kapitel des Befehlssatzes beschrieben, werden die 32-Bit-Befehle im Decoder in die einzelnen Informationsbausteine zerlegt.

\begin{figure}[h]
\centering
% \includegraphics{Figure 3.2.1}
\caption{Bild}
\end{figure}

In der ersten Version des Decoders verglich der Decoder den ermittelten \enquote{Opcode} mit dem des Befehlssatzes.
Dies wurde durch eine \enquote{Fallunterscheidung}(cases) realisiert. Nachdem eine Übereinstimmung gefunden wurde, wurde das Immediate-Bit geprüft.
In Abhängigkeit des \enquote{Opcodes} und des Immediate-bits wurden dann die weiteren Informationen des 32Bit-Codes vom Decoder an die entsprechenden Outputs weitergeleitet. 
Dies sind:

\begin{description}
 \item [Alu-Opcode]
 Der Alu-Opcode ist ein 5-Bit großer Befehlssatz für die Alu.
 Er wurde eingeführt, um gleiche Verhaltensweisen der Alu, bei unterschiedlichen "Opcodes" zusammenzufassen.

\item [Register]
 Hierbei handelt es sich um Register, die entweder gelesen oder beschrieben werden sollen.

\item [Immediate-wert]
Der Immediate-Wert ist eine 16-Bit lange Zahl, die als Adresse oder Inhalt interpretiert werden kann.

\item [Write-Enable]
 Vergleichbar mit einer Flag, ist das \enquote{Write-Enable} ein 1-Bit großes Signal, das ausschließlich dafür benötigt wird, um zu bestimmen, ob ins Register geschrieben werden soll.
\end{description}

Um die Decodierung effizienter zu gestalten, wurden die ganzen \enquote{Fallunterscheidungen} der einzelnen \enquote{Opcodes} durch eine Unterscheidung ihres \enquote{Layouts} ersetzt.
Dadurch werden nun nicht alle 5 Bits des \enquote{Opcodes} überprüft, sondern nur die ersten 2 Bits.
Diese 2 Bits sind wie im Kapitel \ref{ch:Befehlssatz} beschrieben, ausschlaggebend für das Layout des 32-Bit-Codes.
% Neu ab hier
In Abhängigkeit des \enquote{Layouts} und des Immediat-Bits werden immer bestimmte Bits des 32-Bit-Codes an die Outputs zugewiesen.


\begin{figure}[h]
\centering
% \includegraphics{Figure 3.2.1}
\caption{Bild}
\end{figure}

*Ausnahme ist der Move-Befehl, da diese aus inhaltlicher Logik anders zu behandeln ist.

Je nach Layout sind somit andere Bits des 32-Bit-Codes relevant bzw. andere irrelevant. Es wird somit nicht geprüft, was sich genau in den irrelevanten Teilen des Codes befindet. Ähnlich wie bei einem KV-Diagramm, könnte man diese als  \enquote{don't care} behandeln. Dadurch wird die Logik im Decoder minimiert. 
Ein weiterer Vorteil dieser Entscheidung ist, dass weitere Befehle so einfacher zu ergänzen sind, da sie nur der Grammatik des dazugehörigem Layout entsprechen müssen um deren Inhalt zu dekodieren.  


(paddy)

\section{ALU}

Alu

Die ALU ist für die logischen und arithmetischen Operationen verantwortlich.
Sie erhält ihren Opcode vom Decoder.
Es gibt zwei Operanden als Input und ein Input Flag - das Carry Bit.
Herzstück der ALU ist eine große Switch Anweisung, die abhängig vom Opcode die zwei Operanden mit einander verrechnet.
Die ALU unterstützt insgesamt 14 Operationen. 
Die Overflow, CarryOut und Compare Flags, werden konstant berechnet.
Ursprünglich wurde das Flag-Register jeden Taktzyklus geschrieben.
Die ALU \enquote{änderte} Flags nur, sofern es nötig war.
Dies führte zum gleichen Problem, wie auch beim PC Handling.
Nachfolgende Befehle erhielten noch die alten Flagwerte.
Da bei jedem Takt ins Flagregister geschrieben wurde, überschrieb der nachfolgende Befehl die gesetzten Flags des vorherigen.
Die Lösung bestand darin, die Carry- und Overflowflags nur bei Addition und Subtraktion zu schreiben.
Das Compare-flag wurde nur entsprechend gesetzt, sofern ein Compare-befehl in write-back war.
Das Overflow-Flag ist nicht implementiert.
Es lag bisher nicht im Fokus.
Die Zielberechnung von Fibonacci funktioniert auch ohne.



(freddy)

\chapter{Software}
% chapter introduction missing - brauchen wir die hier wirklich? -> siehe Chapter 4.1

(lennart)

\section{Anforderungsanalyse}
Zu Beginn des Softwareentwicklungsprojektes bestand die erste Herausforderung darin, eine auf das Problem zugeschnittene Anforderungsbeschreibung zu erstellen.
Die grundsätzlichen Anforderungen konnten dabei prinzipiell in die funktionalen und die technischen Anforderungen unterteilt werden. 

\subsection{Funktionale Anforderungen}
Die funktionalen Anforderungen beschreiben, über welche Funktionen die Software am Ende des Entwicklungsprozesses verfügen muss.
Um eine realistisch zu bewältigende Menge an Funktionen auswählen zu können, war eine  kritische Selektion dieser notwendig. 
So gab es z.B. zu Beginn die Idee eine diskrete Simulationsumgebung zu erstellen. Allerdings wurde diese Möglichkeit wieder verworfen, da unklar war, wie lange die Implementation dieses Features gedauert hätte und wie lange die Entwicklung des Compilers selbst dauert.
Wir entschieden uns also zunächst dafür, uns ausschließlich auf den Compiler (Assemblercode zu Binärcode) zu fokussieren.
Die dafür notwendigen funktionalen Anforderungen konnten relativ schnell ermittelt werden.

\subsubsection{Primäre Funktionen}
Zum einen bestand das primäre Ziel darin, eine beliebige Textdatei mit Assemblercode automatisch einlesen zu können, aus dem eine neue Textdatei - bestehend aus Binärcode - generiert werden sollte. Zum anderen musste es eine geeignete Möglichkeit geben syntaktische Fehler innerhalb des Assemblercodes zu erkennen und auszugeben.

\subsubsection{Sekundäre Funktionen}
Neben den primären Funktionen sollte es die Möglichkeit geben innerhalb der Software Makros zu erstellen, welche Operationen ermöglichen, die nicht im Instruction Set explizit definiert sind. Beispielsweise besitzt unser Instruction Set keinen Multiplikationsbefehl (\texttt{MUL}).
Dennoch kann der \texttt{MUL}-Befehl im Assemblercode benutzt werden, da er intern durch die russische Bauernmultiplikation ersetzt wird.
Zusätzlich dazu war unklar, inwieweit das Hardwareentwicklungsteam auf Hazards im Rahmen des Pipelining reagieren könnte. Aus diesem Grund musste ein softwaretechnischer Lösung  gefunden werden, welche auf jedenfall funktioniert.
Zu Beginn wurde lediglich festgelegt, dass wir nach jedem Befehl eine notwendige Anzahl NOOP-Befehle einfügen.
Da dieses Prinzip allerdings völlig ungeeignet ist - wofür eine Pipeline benutzen, wenn sie obsolet gemacht wird? - entschieden wir uns dazu bestimmte Abhängigkeiten von Beginn an mit ein zu beziehen.
Z.B. befüllen der Pipeline vor einem Jump-Befehl mit ausreichend vielen NO-Ops oder das Erkennen der Abhängigkeiten von genutzten Registern.

\subsection{Technische Anforderungen}
Nach Fertigstellung der funktionalen Anforderungsanalyse gingen wir dazu über, eine Liste an technischen Anforderungen des Systems zu entwerfen.

\subsubsection{Änderbarkeit}
Aufgrund des (vorläufigen) Mangels an Fachkenntnis war es von sehr großer Bedeutung das System so zu gestalten, dass Änderungen an der Struktur des Assemblercodes oder an den funktionalen Anforderungen ohne großen Aufwand gemacht werden konnten.

\subsubsection{Erweiterbarkeit}
Da zu Beginn des Entwicklungsprozesses nur schwer abzuschätzen war, wie lange wir für die Implementation von bestimmten Funktionen brauchen, war es notwendig das System so zu gestalten, dass eine einfache Erweiterbarkeit von Funktionalitäten gegeben ist. Wir entschieden uns daher ein geeignetes Framework zu suchen, welches dieses Qualitätsmerkmal ausreichend erfüllt. 

\subsubsection{Korrektheit}
In Folge der Tatsache, dass Binärcode nur äußerst schwer zu Debuggen ist, war eine Hauptanforderung ein hohes Maß an Korrektheit der Übersetzung.
Aufgrund der anfänglich ungünstigen Designentscheidung keinen diskreten Simulator zum Testen zu entwickeln, mussten wir warten, bis das Hardwareentwicklungsteam einen ausreichend funktionierenden Simulator fertiggestellt hatte. Aus diesem Grund dauerte das Debugging wesentlich länger als notwendig.

\subsubsection{Vernachlässigte Qualitätsmerkmale}
Standardmäßig gibt es im Laufe einer jeden Softwareentwicklung Entscheidungen zu signifikanten Qualitätsmerkmalen zu fällen.
So entschieden wir uns bewusst dafür, dass die Effizienz des Programms eine sehr untergeordnete Rolle spielen sollte, da es praktisch ausgeschlossen war, dass wir große Mengen an Quellcode in sehr kurzer Zeit übersetzen müssen.

(leonard)

\section{ANTLR}
%https://www.antlr.org/
In der Vorbereitung für die Entwicklung des Compilers haben wir verschiedene Möglichkeiten abgewogen, wie die von uns gesteckten Ziele am besten zu erreichen sind.
Die simpelste Möglichkeit wäre es gewesen den Quellcode direkt mit einem großen Switch Statement zu parsen und die einzelnen Befehle dann zu übersetzen.
Da dies jedoch schnell relativ unübersichtlich zu werden schien und eine Kontradiktion zu den gesetzten funktionalen Anforderungen gewesen wäre, haben wir Ausschau nach möglichen Tools gehalten, mit denen wir unsere Ziele effizienter erreichen.
Dabei sind wir auf ANTLR gestoßen.
ANTLR (ANother Tool for Language Recognition) ist ein Parser Generator zum Lesen und Verarbeiten von strukturierten Textdateien auf der Basis von Grammatiken, der unter anderem verwendet werden kann, um eigene Programmiersprachen zu entwickeln.
% cite ANTLR
Um einen Compiler mit ANTLR zu erstellen, benötigt man zunächst eine Grammatik, in der die formalen Regeln der Sprache beschrieben werden.
Die Grammatik besteht aus Parser- und Lexer-Regeln.
Dabei stehen die Parserregeln für die Struktur der Sprache (Nonterminale) und die Lexerregeln für die tatsächlichen Zeichen/Wörter (Terminale), die die Nonterminale ersetzen können.
Die Grammatik kann mit Hilfe von ANTLR zu Java Klassen compiliert werden.
ANTLR liefert dabei einen Lexer und einen Parser mit deren Hilfe ein Parsetree generiert werden kann.
Dazu wird einfach eine Datei mit dem Quellcode eingelesen und mithilfe von Lexer und Parser verarbeitet.
Der resultierende Parsetree kann mit Hilfe von Visitors durchlaufen werden,
welche dann die eigentliche Übersetzungsarbeit leisten.

(lennart \& leonard)

\section{Compiler Implementation}

\subsection{Grammatik}
ANTLR nutzt zum Parsen eine kontextfreie Grammatik der Form: $G = (V, T, P, S)$. Hierbei entspricht V der endlichen Menge der Variablen (Menge der Ableitungsregeln, bzw. Parser-Regeln), T der Menge der Terminalen (Tokens, bzw. Lexer-Regeln), P den Produktionsregeln (Parser-Regeln bzw. Ableitungsregeln) und S der Startregel. Basierend auf der Grammatik existiert also eine Sprache L(G). Für jeden von uns vorgesehenen Assemblerbefehl A gilt also: $A \in L(G)$. 

(leonard)

\subsubsection{Lexer - Lexikalischer Scanner}
Der Lexer stellt die unterste Ebene einer Grammatik dar.
Jede Lexer-Regel definiert dabei genau einen Token.
Ziel des Lexers ist es nun für jede mögliche Eingabe zu erkennen, um welches Token es sich handelt.
Hierbei gilt zu beachten, dass ein Token nicht zwangsläufig exakt definiert werden muss, da die Nutzung von regulären Ausdrücken möglich ist.
%listng austauschen, kommaenatre weg, in eine fig. rein 
\begin{lstlisting}
MOV: 'MOV';
BINARY: '0b' ([0-1])+;
\end{lstlisting}
Zu Beginn des Prozesses werden so zunächst sämtliche Eingaben in Tokens umgewandelt, welche dann von den Parser-Regeln in einen Syntax-Baum überführt werden.

(leonard)

\subsubsection{Parser}
Die Parser-Regeln definieren den Ableitungsbaum der formalen Grammatik.
Dabei entspricht jede Verzweigung einer Parser-Regel, während jeder Blattknoten einem Token entspricht.

\begin{figure}[h]
\centering
% \includegraphics{Figure 3.2.1}
\caption{Syntaxbaum des Befehls: \texttt{MOV, r10, 10;}}
\end{figure}

Eine Parser-Regel hat die Form
%listng austauschen, kommaenatre weg, in eine fig. rein 
\begin{lstlisting}
start: start command | command;
\end{lstlisting}
wobei die Regel entweder direkt ein Token nutzt oder auf eine weitere Regel verweist.
Anhand der Ableitungsregeln kann also eine Struktur erstellt werden, auf welche nun spezifisch reagiert werden kann.
So ist es innerhalb des Programmes nun möglich auf jeden Knoten des Baumes, sowie seine Eltern- und Kindknoten zuzugreifen.

(leonard)

\subsection{Makros}
Da unser Compiler mehrere Aufgaben erfüllen soll, die nur sequentiell erledigt werden können, wird dieser Ablauf insgesamt drei mal durchlaufen.
Bei jedem Durchlauf wird ein anderer Arbeitsschritt erledigt.
%eventuell die 3 punkte in anführungszeichne
Die drei Schritte sind das Ersetzen von Makros durch ihre jeweilige Implementation, das Einfügen von NOOPS zur Vermeidung von Hazards und die eigentliche Übersetzung.
Makros wurden von uns mit eigenen Lexer- und Parser-Rules innerhalb der Grammatik implementiert.
Dies ermöglicht es uns die so definierten Makros wie normale Befehle im Quellcode zu verwenden.
Der Compiler kümmert sich dann darum die Mnemoniks durch entsprechende Funktionen zu ersetzen.
% erläutern: mnemoniks (vllt in ner fussnote)
Zum aktuellen Zeitpunkt haben wir die russische Bauernmultiplikation für positive Zahlen (MUL) und ein simples binäres Invertieren (NOT) durch Makros verwirklicht.
Es lassen sich leicht zusätzliche Makros hinzufügen, indem diese in der Grammatik definiert  und entsprechende Implementation im Compiler verwirklicht werden.
Bei der Verwendung von Makros muss beachtet werden, dass diese nicht immer \enquote{in place} implementiert werden können.
% was meint in place? erläútern.
Es ist daher empfehlenswert, zu dokumentieren, welche Register von den jeweiligen Makros verwendet werden, um Inkonsistenzen zu vermeiden.
Das Vorgehen des Makro-Visitors besteht also darin den übergebenen Quellcode größtenteils unverändert zurück zu geben und nur die Stellen, an denen ein Makro entdeckt wird, zu ersetzen.

(lennart)

\subsection{Jumplabels und NOOPS}
Im zweiten Übersetzungsschritt werden die jump-Labels zu absoluten Zeilenangaben übersetzt und NOOPS eingefügt.
Die Sprungadressen werden ermittelt, indem der Compiler mitzählt, wie viele Befehle bereits gesehen wurden.
%"gesehen werden" gut formuliert
Da Sprünge im Instruktionsspeicher über einen Befehlsindex implementiert sind, reicht dieser aus um Sprünge mit Immediate zu realisieren.
Die Indizes werden zusammen mit den Labels gespeichert, sodass die Labels im Übersetzungsschritt durch die Immediate Werte ersetzt werden können.
Eine weitere Aufgabe innerhalb dieses Schrittes ist das Einfügen von NOOPS, um Hazards zu vermeiden.
Um dies zu bewerkstelligen, merkt sich der Compiler für jedes Register die Anzahl der notwendigen NOOPS, die eingefügt werde müssten, falls das Register im aktuellen Befehl vorkommt.
Ist der Wert 0 wird für das jeweilige Register der Wert auf 4 gesetzt, was der Anzahl der Pipelinestufen entspricht.
Falls der Wert nicht 0 ist, werden entsprechend viele NOOPS eingefügt.
%was heißt entsprechend viele ? 
%anmerken dass nach jedem Schritt dekrementiert wird
Außerdem werden generell nach jedem Sprung (\texttt{JMP}, \texttt{B}) 3 NOOPS eingefügt, da wir für den Prozessor keinen Flush implementiert haben.
% Warum nur 3 nicht 4??
Auf diese Weise wird vermieden, dass ungewollte Befehle in die Pipeline geladen werden.

(lennart)

\subsection{Übersetzung}
Im letzten Schritt wird der auf diese Weise modifizierte Quellcode in den Maschinencode übersetzt.
Die einzelnen Befehle werden je nach Befehlstyp (NOOP, TWOOP, THREEOP, Jump) unterschiedlich behandelt.
Die Opcodes stehen in einer Hashmap zusammen mit den Mnemoniks für die einzelnen Befehle.
Ist der letzte Operand ein Immediate Wert wird das Immediate-Bit wird entsprechend gesetzt.
Die Immediates werden aus dem angegebenen Zahlensystem (binär, hexadezimal, dezimal) zu einer Binärzahl übersetzt.
Alle Operanden werden je nach Befehlsstruktur mit einem entsprechendem Padding versehen.
%erklären was ein padding ist, mehr ausführen, bild bei befehlswörtern?
Anschließend wird das Befehlswort zusammengesetzt und zusammen mit einem führenden Befehlsindex zurückgegeben.
%was ist ein führender Befehlsindex

%\subsection{Probleme}
%Die Implementation des Compilers verlief relativ Problemlos und das Ergebnis ermöglicht uns alle von uns gesetzten Ziele zu erreichen. Eine Schwäche unseres Vorgehens war jedoch die schlechte Testbarkeit des Programms, da diese von der Implementation der Hardware abhängig war. Das führte dazu, dass Fehler erst spät sichtbar wurden, da wir den compilierten Quellcode erst an dem fertigen Prozessor testen konnten (siehe Kapitel Test).
%Simulator hätte das lösen können

(lennart)

\chapter{Ergebnis}

% Gleicher ablauf wie bei Vortrag
% Prozess von assembler zu maschinencode zu cpu, mit "fertigem" fibonnaci im ram
% Anwedung des beschrieben codes (hardware wie software) zeigen.
% -> Verweis auf assembler / machinencode

% zu software padding paste nicht
% zu software anzahl noops zu ändern


\begin{itemize}
    \item Wie lief das beim Fibonacci Test
    \item Padding passte nicht
    \item Nicht passende NOOPS
    \item manuelles Fixen des Maschinencodes
    \item Problem mit Carry-flags (Durchschleifproblem)
    \item Problem mit Jumpaddressen (Durchschleifproblem)
\end{itemize}

Unser next\_seq\_pc ansatz war nicht ganz durchdacht.
Die Idee war den nächsten PC in der Fetch stage schon zu bestimmen und \enquote{durchzuschleifen} bis zur Verwendung in memory access.
Dort wird dieser verwendet um die Instruction im fetch zu bestimmen.
Wenn nun in allen Pipelineregistern dieser next\_seq\_pc mit 0 initialisiert wird, fetchd der prozessor auch 4 mal die instruction an der Stelle 0.
Da die fetch stage auch auf basis dieses Wertes den nächsten next\_seq\_pc berechnet hat (in diesem Fall 1) ist die instruction für die nächsten 4 instruction die an Adresse 1.

Dieses Problem zeigte sich auch bei den Flags und bei den Jumpaddressen.
Wenn ein Flag gesetzt werden sollte, setze die ALU dieses flag, wenn nicht wurde der vorherige Flagwert gesetzt.
Da das \enquote{setzen} in der ALU und das schreiben im Writeback einige Tacktzyclen ausseinander lag, lass der nachfolgende Befehl die \enquote{alten} Flags und setzte diese im Writeback direkt wieder.
Somit war es nicht möglich flags, wie zB. für compare / branch operation notwending zu setzen.
Jumps waren von diesem Durchschleifproblem ebenso betroffen. Ein neu gesetzer PC wurde von den nachfolgenden next\_seq\_pc Werten wieder überschrieben.

Um dies zu beseitigen änderten wir die Flag und PC logik komplett.
Das in der mem access stage verwendete Flag flags\_comp um ein branch auszuwerten wird im inst decode gelesen und bis zum mem access durchgeschliffen.
Es ist unabhängig von den von der alu gestetzten flags, die bis zum writeback durchgeschliffen wird.
Dadurch müssen zwar mehr NOOPS eingeführt werden zwischen compare und branch aber beide \enquote{Seiten} des flag setzens sind sauber getrennt.

Der PC wird nun \enquote{zentral} gehalten.
Es gibt ihn nurnoch einmal.
Die Writeback stage entscheidet nun mithilfe eines \enquote{will\_jump} signals, ob gesprungen wird und passt diesen bei Bedarf an.
Wenn nicht gesprungen wird, erhöht die write back stack den PC um 1.

Ein weiteres Issue, dass während der ersten Tests auftrat, war die Anzahl der NOOPS.
Es brauchte einiges herumprobieren um heraus zu finden, wie viele NOOPS wir bei jedem Befehl benötigen.

In unserem finalen Versuch war es uns möglich, ein Fibonacci-Programm ablaufen zu lassen.

In früheren Überlegungen fiel uns auf, dass ein SUB nichts anderes ist als ein ADD mit einem Operant bitweise invertiert.
Deshalb war SUB als ADD aluop implementiert.
Wir vergaßen dabei einen Operanden zu invertieren.
Dies lies sich recht leicht korrigieren.
Die ALU kennt nun die Operation SUB.

(max)

\subsubsection{Ergebnis - Softwareentwicklung}
Einer der schwierigsten und gleichzeitig weitreichendsten Entscheidungen war die Wahl des genutzten Frameworks. Zu Beginn des Softwareentwicklungsprozesses mussten wir zunächst viel Zeit in diese Entscheidung investieren, da es (neben ANTLR) viele verschiedene Parser-Generatoren mit sehr unterschiedlichen Ansätzen und damit verbundenen Vor- und Nachteilen gibt. Es kostete uns relativ viel Zeit hierbei den für uns richtigen Auszuwählen. %Das ist ein bisschen gelogen...
Als weiterer aufwändiger Schritt während der Entwicklung stellte sich das Erlernen und die korrekte Anwendung von ANTLR heraus. ANTLR bietet ein äußerst vielfaltiges Spektrum an nutzbaren Funktionen und Möglichkeiten, was sowohl die Grammatik als auch die Implementation angeht. Ein schwierige Teil hierbei war nun herauszufinden, welche Funktionen tatsächlich gebraucht würden und welche vernachlässigbar waren. %absatz
Abschließend lässt sich sagen, dass die release Version der Software alle in Kapitel 4.1 %verlinkung zu Chapter 4.1
genannten funktionalen und technischen Anforderungen besitzt. Dies betrifft sowohl die Bedienung  der Software als auch eine potenzieller Weiterentwicklung bzw. das Hinzufügen von neuen Features. 

(leonard)


\chapter{Konklusion und Ausblick}
\begin{itemize}
    \item HLT op
    \item Refactoring (unused signals)
    \item Overflow out
    \item link register is wrong
    \item Reset Signal / Initialization
    \item Pipeline flushing
\end{itemize}

% bewertung eigener Prozess
% Software: messen an gesetzten masstäben
% Reflektiv sein
% Bewertung des ergebniss
% Vergleich zu vorher / planungsphase

Es sind noch viele Möglichkeiten offen geblieben den Prozessor und den Assembler-Compiler auszubauen.
Zum Einen fehlen noch Implementationen für eine Opcodes. %bezieht sich das auf den Assembler-Compiler oder den Prozesser?
Zum Anderen sind der Overflow der ALU und das Link Register nicht korrekt implementiert.

Um den Prozessor synthetisieren zu können müssen mindestens Signale richtig initialisiert werden.
Zur Zeit wird in VHDL ein Anfangswert gesetzt.
Hier wäre es nötig ein Reset-Signal einzuführen, dass den PC auf 0 und die Pipeline Register Opcodes auf NOOP setzt.
Auch Befehle wie COMPG nicht gut umgesetzt.
In VHDL lässt sich die dafür notwendige Operation mit einem simplen \enquote{>} umsetzten.
Dies wird aber in der Synthese eine komplexe Schaltung erzeugen.

Der Prozessor im Zusammenspiel mit dem Assembler-Compiler kann aus seiner Pipelinestruktur noch nicht allzu viel herausholen.
Alle Abhängigkeite, die zu Hazards führen können, müssen eine entsprechende Anzahl von NOOPS nach sich ziehen.
Würde die CPU Pipeline Flushing implementieren ließe sich hier einiges an NOOPS sparen.
Man könnte noch einen Schritt weiter gehen und Instruction Reordering oder auch \enquote{Backpropagation} von Ergebnissen betreiben.

Eine HLT op wäre noch schön.

(max)

\subsubsection{Konklusion \& Ausblick - Softwareentwicklung}
Basierend auf unseren Ergebnissen lässt sich sagen, dass unser Vorgehen und die Wahl des Frameworks die richtige war. 
So stellte sich unsere anfänglichen Entscheidung eine vollständige Anforderungsanalyse zu entwickeln und insbesondere ANTLR zu nutzen im Nachhinein als sehr gut heraus, denn es führte dazu, dass im Verlauf der Entwicklung des Parsers keine nennenswerten Komplikationen auftraten. Dies wurde zusätzlich dadurch begünstigt, dass sich die grundsätzlichen funktionalen und technischen Anforderungen an die Software während des Entwicklungsprozesses nicht änderten.

Allerdings gibt es noch eine Menge an zusätzlichen Features, welche für eine professionelle Entwicklung unter Anwendung unseres Tools notwendig wären. Z.B. müsste es entweder eine diskrete Simulationsumgebung geben, mit der man seinen Quellcode direkt testen kann oder es müsste eine direkte Schnittstelle zu VHDL %Hier nochmal mit der Hardware-Gruppe reden! 
entwickelt werden. Zusätzlich dazu müsste es im Rahmen der Simulationsumgebung einen Debugger geben um Probleme während der Laufzeit vernünftig behandeln zu können.
Da Assemblercode nicht mehr zeitgemäß ist, müsste man für komplexere Programme einen Schritt weiter gehen und einen Compiler für eine existierende Hochsprache (z.B. C) implementieren oder eine eigene Hochsprache Entwickeln. Um mit einer selbst entwickelten Hochsprache vernünftig arbeiten zu können, wäre hierfür allerdings ein GUI notwendig. %Warum braucht man eine GUI?

(leonard)

%\kant[10-14]

\end{document}

% software: konklusion befüllen
% software: stuff zur konklusion bewegen
% software: erlernen von antlr war komplex - Grammatik + Klassenhierarchie
% hardware: input/output - probleme und lösung skizzieren
% hardware: register - probleme und lösung skizzieren
% hardware: pc handling - probleme und lösung skizzieren
% hardware: decoder - probleme und lösung skizzieren
% hardware: alu - probleme und lösung skizzieren
% hardware: vorwort/vorgehen und struktur "zusammenschreiben"