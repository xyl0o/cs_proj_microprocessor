\documentclass[paper=a4,fontsize=10pt]{scrreprt}

\usepackage{fontspec}
\usepackage{polyglossia}
\setmainlanguage[babelshorthands=true]{german}

\usepackage[autostyle,german=quotes]{csquotes}
\usepackage[autostyle]{csquotes}

\usepackage[toc,page]{appendix}

\usepackage{array}
\usepackage{multirow}
\usepackage{multicol}
\usepackage{hyperref}
\usepackage{url}

\usepackage{graphicx}

\usepackage[
    backend=biber,
    style=numeric,
    citestyle=authoryear,
]{biblatex}

\addbibresource{./literature.bib}

\usepackage{amsmath}

\title{Project::Foo Abschlussbericht}
\author{Maximilian Bauregger \and Leonard Caanitz \and Lennart Clasmeier \and Patrizio Ferrara \and Luca Müller \and Frederic Voigt}
\date{\today}

\begin{document}

\maketitle

\section{Abstract}

\begin{itemize}
    \item Was wurde wie erreicht?
    \item Wo pflegt sich das in den den aktuellen stand der Forschung ein - wir haben also eine Reduzierte Arm Version gebaut
\end{itemize}


\chapter{Einführung}
\begin{itemize}
    \item Hier wird das Projekt kurz vorgestellt
    \item Was haben wir gemacht? Was haben wir vor? Welche Tools wurden benutzt?
    \item Welche Schwierigkeiten haben sich abgezeichnet?
    \item welche Struktur hat der Vortrag - Welche Informationen können die Leser\_Innen in welchen Kapiteln erwarten.
\end{itemize}

\chapter{Planungsphase}

\section{IS}

\begin{itemize}
    \item 32 Bit Befehlswortlänge
    \item 32 Register
    \item 3 Addressmaschine
    \item reduzierter ARM Befehlssatz
    \item Pipelining
    \item Int als einziger Datentyp
    \item Immediate Bit
    \item Relative Jumps
    \item Relative Speicheraddresssierung
    \item 0 register
    \item Python Sketch als Struktur
\end{itemize}

// appendix full instruction set

\section{Diskussionen}

\subsection{16 bit vs. 32 bit vs. 64 bit}
Wir haben uns für 32 bit entschieden, warum?

\subsection{5bit opcode vs 6bit opcode}
Wir haben 5opcode mit nem zusätzlichen immediate bit, warum?

\subsection{3 Addressmaschine? abhängig von der bit-Zahl}
32 bit erlauben uns 3 address op codes

\subsection{Relative Jumps (nur immediate jumps?)}
Wir haben uns für relative und absolute jumps entschieden (JMP ist ein twoop).

\chapter{Software}

\section{Diskussion}
\begin{itemize}
    \item
\end{itemize}

\section{ANTLR}
\subsection{Grammatik}
\begin{itemize}
    \item Einführung
    \item Lexer
    \item Parser
\end{itemize}

\section{Compiler Implementation}
\subsection{ANTLR - Implementierung}
\begin{itemize}
    \item Automatisch erzeugte Klassen
\end{itemize}

\subsection{Algorithmus}
\begin{itemize}
    \item Macros ersetzen
    \item Nops und Labels parsen
    \item Übersetzen
\end{itemize}

\chapter{Hardware}

\begin{itemize}
    \item VHDL
    \begin{itemize}
        \item Struktur der Komponenten
        \item Zu "wörtliche" Umsetztung des Python-Sketches (alles in entities verpackt, wir wussten nicht wie Register funktionieren)
        \item Probleme mit dem reinen Entity - Architecture - Ansatz
        \item Parallelität und sequenzielle Abläufe
        \item Finaler Entwurf
    \end{itemize}
    \item Tests
    \item Nice additions
\end{itemize}

\chapter{Test}
\begin{itemize}
    \item Wie lief das beim Fibonacci Test
    \item Padding passte nicht
    \item Nicht passende noops
    \item manuelles Fixen des Mashinencodes
    \item Problem mit Carry-flags (Durchschleifproblem)
    \item Problem mit Jumpaddressen (Durchschleifproblem)
\end{itemize}

\chapter{Konklusion und Ausblick}
\begin{itemize}
    \item HLT op
    \item Refactoring (unused signals)
    \item Overflow out
    \item link register is wrong
    \item Reset Signal / Initialization
    \item Pipeline flushing
\end{itemize}

\end{document}
